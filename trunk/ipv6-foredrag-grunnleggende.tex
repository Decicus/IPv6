% $Ximalas$
%\documentclass{beamer}
%\documentclass[handout]{beamer}
%\usepackage{pgfpages}
%\pgfpagesuselayout{2 on 1}[a4paper,border shrink=5mm]
%\pgfpagesuselayout{4 on 1}[a4paper,landscape,border shrink=5mm]
%\setbeameroption{show notes} % on second screen}

\def\jobname{ipv6-foredrag-grunnleggende} % Sett riktig navn p� presentasjonen her.
\setjobnamebeamerversion{\jobname}

\usepackage[cp1252]{inputenc}
\usepackage[T1]{fontenc}
\usepackage[norsk]{babel}
\usepackage{booktabs}

%\usetheme{Antibes}
%\usetheme{Berkeley}
%\usetheme{Berlin}
%\usetheme{Copenhagen}
%\usetheme{Darmstadt}
%\usetheme{Dresden}
%\usetheme{Frankfurt}
%\usetheme{Goettingen}
%\usetheme{Hannover}
%\usetheme{Ilmenau}
%\usetheme{JuanLesPins}
%\usetheme{Luebeck}
\usetheme{Malmoe}
%\usetheme{Marburg}
%\usetheme{Montpellier}
%\usetheme{PaloAlto}
%\usetheme{Singapore}
%\usetheme{Szeged}
%\usetheme{Warsaw}

\hypersetup{colorlinks,linkcolor=,urlcolor=blue}

\title{\textbf{IPv6-foredrag}}
\subtitle{Grunnleggende}
%\author{\href{http://fig.ol.no/~trond/}{Trond Endrest�l}}
\author{Trond Endrest�l}
%\institute{\href{http://fagskolen-innlandet.no/}{Fagskolen Innlandet}}
\institute{Fagskolen Innlandet}
\subject{Emne for bruk i egenskapene til PDF-fila}
\keywords{N�kkelord for bruk i egenskapene til PDF-fila}
\date{\today} % eller
%\date{18.\ september 2013}

\begin{document}

\begin{frame}
  \titlepage
\end{frame}

\begin{frame}
  \frametitle{Foredragets filer}
  \begin{itemize}
  \item Filene til foredraget er tilgjengelig gjennom:
    \begin{itemize}
    \item Subversion: \texttt{svn co \textbackslash\ \url{svn://svn.ximalas.info/ipv6-foredrag-grunnleggende}}
    \item Web: \href{http://svnweb.ximalas.info/ipv6-foredrag-grunnleggende/}{\texttt{svnweb.ximalas.info/ipv6-foredrag-grunnleggende/}}
    \end{itemize}
  \item Hovedfila b�rer denne identifikasjonen:\\
    \texttt{\$${}$Ximalas${}$\$}
  \item Foredraget er mekket ved hjelp av
    \href{http://www.gnu.org/software/emacs/}{GNU~Emacs},
    \href{http://www.gnu.org/software/auctex/}{AUC\TeX},
    \href{http://miktex.org/}{MiK\TeX}, dokumentklassa
    \href{https://bitbucket.org/rivanvx/beamer/wiki/Home}{beamer},
    \href{http://subversion.apache.org/}{Subversion},
    \href{http://tortoisesvn.net/}{TortoiseSVN} og
    \href{http://get.adobe.com/no/reader/}{Adobe Reader}
  \end{itemize}
\end{frame}

%\section*{Oversikt}
%\begin{frame}
%  \tableofcontents
%\end{frame}

\section{Kort om IPv6}
\subsection{Hva er IPv6?}
\begin{frame}
  \frametitle{Kort om IPv6}
  \framesubtitle{Hva er IPv6?}
  \pause
  \begin{itemize}[<+->]
  \item En lag-3-protokoll ment � erstatte IPv4
  \item Har eksistert siden desember 1995, \href{http://tools.ietf.org/html/rfc1883}{RFC~1883}
  \item Enkel grunnheader med fast lengde
  \item Flere utvidelsesheadere, riktig rekkef�lge er viktig
  \item 128-bit adresser
  \item Ny versjon av ICMP: ICMPv6
  \item ARP og RARP for IPv6 er en del av ICMPv6
    \begin{itemize}[<+->]
    \item Ikke n�dvendig med ekstra lim for adressene i lagene 2 og 3
    \end{itemize}
  \item Ny versjon av DHCP: DHCPv6
  \item Automatisk adressekonfigurasjon \textit{uten\/} bruk av DHCPv6
  \end{itemize}
\end{frame}

\begin{frame}
  \frametitle{Kort om IPv6}
  \framesubtitle{Hva er IPv6?}
  \pause
  \begin{itemize}[<+->]
  \item Totalt antall IPv6-adresser:
  \item $2^{128}=340.282.366.920.938.463.463.374.607.431.768.211.456$
  \item Bare $1/8$ kan brukes til offentlige unicast-adresser:
  \item $2^{125}=\phantom{0}42.535.295.865.117.307.932.921.825.928.971.026.432$
  \item Fortsatt mye mer enn det fullstendige IPv4-adresserommet:
  \item $2^{32\phantom{0}}=\phantom{000.000.000.000.000.000.000.000.000.00}4.294.967.296$
  \item Bare $3.702.258.688$ IPv4-adresser kan bli brukt som offentlige IPv4-unicast-adresser
  \item Se Tronds utregning fra 2012:
    \texttt{\url{http://ximalas.info/2012/07/20/how-many-ipv4-addresses-are-there/}}
  \end{itemize}
\end{frame}

\subsection{Hvorfor trenger vi IPv6?}
\begin{frame}
  \frametitle{Kort om IPv6}
  \framesubtitle{Hvorfor trenger vi IPv6?}
  \pause
  \begin{itemize}[<+->]
  \item Verden g�r tom for offentlige IPv4-adresser
  \item \href{http://www.iana.org/}{IANA} gikk tom i \href{http://www.icann.org/en/news/press/releases/release-03feb11-en.pdf}{februar 2011}
    \begin{itemize}[<+->]
    \item \href{http://www.apnic.net/}{APNIC} gikk tom i \href{http://www.apnic.net/community/ipv4-exhaustion/graphical-information}{april 2011}
    \item \href{http://www.ripe.net/}{RIPE} gikk tom i \href{http://www.ripe.net/internet-coordination/ipv4-exhaustion}{september 2012}
    \item Dersom disse oppf�rer seg pent:
      \begin{itemize}
      \item \href{http://www.lacnic.net/en/web/lacnic/inicio}{LACNIC} kan holde p� til \href{http://www.lacnic.net/web/lacnic/reporte-direcciones-ipv4}{juni 2014}
      \item \href{https://www.arin.net/}{ARIN} kan holde p� til \href{http://www.potaroo.net/tools/ipv4/}{desember 2014}
      \item \href{http://www.afrinic.net/}{AFRINIC} kan holde p� til \href{http://www.potaroo.net/tools/ipv4/}{oktober 2020}
      \end{itemize}
    \end{itemize}
  \end{itemize}
\end{frame}

\begin{frame}
  \frametitle{Kort om IPv6}
  \framesubtitle{Hvorfor trenger vi IPv6?}
  \pause
  \begin{itemize}[<+->]
  \item NAT (\href{http://tools.ietf.org/html/rfc2663}{RFC~2663}), CGN
    (\href{http://tools.ietf.org/html/rfc6264}{RFC~6264}) og Shared
    Address Space
    (\href{http://tools.ietf.org/html/rfc6598}{RFC~6598}) er bare
    st�ttebandasje
    \begin{itemize}[<+->]
    \item Glem det
    \item Ende-til-ende-konnektivitet blir best oppn�dd uten noen
      former for adresseoversettelse
    \end{itemize}
  \item Hierarkisk adressestruktur
  \item Enklere planlegging av subnett sammenlignet med IPv4
    \begin{itemize}[<+->]
    \item De fleste IPv6-subnett bruker et 64-bit prefiks
    \item Autokonfigurasjon \textit{krever\/} et 64-bit prefiks
    \item Fast prefikslengde p� 64 bit er ikke et absolutt krav
    \item DHCPv6 eller manuell konfigurasjon (kan) brukes n�r
      prefikslengda er ulik 64 bit
    \end{itemize}
  \item Kortere rutingtabeller
  \end{itemize}
\end{frame}

\begin{frame}
  \frametitle{Kort om IPv6}
  \framesubtitle{Hvorfor trenger vi IPv6?}
  \pause
  \begin{itemize}[<+->]
  \item Uninett annonserer disse IPv4-subnettene med BGP:
  \item \texttt{78.91.0.0/16},\hfill\texttt{128.39.0.0/16},
      \hfill\texttt{129.177.0.0/16},\hfill\texttt{129.240.0.0/15},
      \hfill\texttt{129.242.0.0/16},\hfill\texttt{144.164.0.0/16},
      \hfill\texttt{151.157.0.0/16},\hfill\texttt{152.94.0.0/16},
      \hfill\texttt{156.116.0.0/16},\hfill\texttt{157.249.0.0/16},
      \hfill\texttt{158.36.0.0/14},\hfill\texttt{161.4.0.0/16},
      \hfill\texttt{193.156.0.0/15},\hfill\texttt{192.111.33.0/24},
      \hfill\texttt{192.133.32.0/24},\hfill\texttt{192.146.238.0/23}
    \item Til gjengjeld trenger Uninett bare � annonsere dette
      IPv6-prefikset:
    \item \texttt{2001:700::/32}
  \end{itemize}
\end{frame}

\subsection{IPv6 ved Fagskolen Innlandet}
\begin{frame}
  \frametitle{Kort om IPv6}
  \framesubtitle{IPv6 ved Fagskolen Innlandet}
  \pause
  \begin{itemize}[<+->]
  \item 1994: Ble tildelt \texttt{128.39.174.0/24} av Uninett
  \item 1.~juni 2005: Ny IT-ansvarlig, yours truly
  \item H�sten 2005: Fikk reservert IPv4-serien
    \texttt{128.39.172.0/23}
  \item P�ska 2006: Fikk reservert IPv6-serien
    \texttt{2001:700:1100::/48}
  \item F�r og etter pinsehelga 2006: Fiberlinjer fra serverrommet og
    til sentralt punkt i hver etasje i hovedetasjen
  \item Sommeren 2006: Nytt Cisco-gear som Catalyst 3560G og 2960
    \begin{itemize}[<+->]
    \item \texttt{128.39.174.0/24} ble brukt til servernett og
      ansattnett, m.m.
    \item \texttt{128.39.172.0/24} ble brukt til datalab
    \item \texttt{128.39.173.0/24} ble brukt til klienter p� tr�dl�st
      studentnett
    \end{itemize}
  \end{itemize}
\end{frame}

\begin{frame}
  \frametitle{Kort om IPv6}
  \framesubtitle{IPv6 ved Fagskolen Innlandet}
  \pause
  \begin{itemize}[<+->]
  \item 6.~september 2006: IPv6-linknettet
    \texttt{2001:700:0:11D::/64} ble aktivert mellom HiG/Uninett og
    FSI
    \begin{itemize}[<+->]
    \item \texttt{2001:700:0:11D::1/64} brukes hos HiG
    \item \texttt{2001:700:0:11D::2/64} brukes hos FSI
    \end{itemize}
  \item Samme dag ble IPv6-subnett innf�rt for FSI-VLAN-ene 20, 30, 70 og 80.
    \begin{itemize}[<+->]
    \item FSI-VLAN 20: \texttt{2001:700:1100:1::/64}
    \item FSI-VLAN 30: \texttt{2001:700:1100:2::/64}
    \item FSI-VLAN 70: \texttt{2001:700:1100:3::/64}
    \item FSI-VLAN 80: \texttt{2001:700:1100:4::/64}
    \end{itemize}
  \item Sommeren 2007:
    \href{http://www.sixxs.net/tools/grh/ula/}{Genererte} og
    \href{http://www.sixxs.net/tools/grh/ula/list/}{registrerte}
    ULA-serien
    \href{http://www.sixxs.net/tools/whois/?fd5c:14cf:c300::/48}{\texttt{FD5C:14CF:C300::/48}}
    for FSI-VLAN som tidligere bare brukte
    \href{http://tools.ietf.org/html/rfc1918}{RFC-1918}-adresser.
  \end{itemize}
\end{frame}

\begin{frame}
  \frametitle{Kort om IPv6}
  \framesubtitle{IPv6 ved Fagskolen Innlandet}
  \pause
  \begin{itemize}[<+->]
  \item H�sten 2010: Enda en IPv4-serie ble innf�rt:
    \texttt{128.39.194.0/24}
    \begin{itemize}[<+->]
    \item \texttt{128.39.172.0/23} brukes til klienter p� tr�dl�st
      studentnett
    \item \texttt{128.39.194.0/24} brukes til datalab etter samme
      m�nster som for \texttt{128.39.172.0/24}
    \end{itemize}
  \item I dag er de fleste brukere kasta over til OFK-nettene
  \item Dette skjedde etter ombyggingen i 2011--2012
  \item Andreklasse data er velsignet med � kunne velge mellom FSI- og
    OFK-nettene
  \item Andreklasse data velger som regel det f�rstnevnte, vanligvis
    FSI-VLAN 48, \texttt{128.39.194.192/27} og
    \texttt{2001:700:1100:8008::/64}
  \end{itemize}
\end{frame}

\begin{frame}
  \frametitle{Kort om IPv6}
  \framesubtitle{IPv6 ved Fagskolen Innlandet}
  \pause
  \begin{itemize}[<+->]
  \item Alle FSI-VLAN har b�de IPv4- og IPv6-adresser
  \item FSI-VLAN med offentlige IPv4-adresser bruker offentlige
    IPv6-adresser fra \texttt{2001:700:1100::/48}-serien
  \item FSI-VLAN med private IPv4-adresser
    (\href{http://tools.ietf.org/html/rfc1918}{RFC~1918}) bruker
    private IPv6-adresser fra \texttt{FD5C:14CF:C300::/48}-serien
  \item Private adresser brukes for alt utstyr som ikke har behov for
    internettforbindelse:
    \begin{itemize}[<+->]
    \item Switcher (med unntak av kjerneswitchen som er L3-router for
      nettverket ved FSI)
    \item Basestasjoner
    \item WLAN-kontroller
    \item UPS-er
    \item Skrivere
    \item VPN-klienter
    \end{itemize}
  \end{itemize}
\end{frame}

\subsection{RFC-er om IPv6}
\begin{frame}
  \frametitle{Kort om IPv6}
  \framesubtitle{RFC-er om IPv6}
  \pause
  \begin{itemize}[<+->]
  \item IPv6-spesifikasjon:
    \href{http://tools.ietf.org/html/rfc2460}{RFC~2460},
    \href{http://tools.ietf.org/html/rfc5095}{RFC~5095},
    \href{http://tools.ietf.org/html/rfc5722}{RFC~5722},
    \href{http://tools.ietf.org/html/rfc5871}{RFC~5871},
    \href{http://tools.ietf.org/html/rfc6437}{RFC~6437},
    \href{http://tools.ietf.org/html/rfc6564}{RFC~6564},
    \href{http://tools.ietf.org/html/rfc6935}{RFC~6935} og
    \href{http://tools.ietf.org/html/rfc6946}{RFC~6946}
  \item ICMPv6: \href{http://tools.ietf.org/html/rfc4443}{RFC~4443} og
    \href{http://tools.ietf.org/html/rfc4884}{RFC~4884}
  \item Neighbor Discovery:
    \href{http://tools.ietf.org/html/rfc4861}{RFC~4861},
    \href{http://tools.ietf.org/html/rfc5942}{RFC~5942} og
    \href{http://tools.ietf.org/html/rfc6980}{RFC~6980}
  \item Path MTU: \href{http://tools.ietf.org/html/rfc1981}{RFC~1981}
  \item DHCPv6: \href{http://tools.ietf.org/html/rfc3315}{RFC~3315},
    \href{http://tools.ietf.org/html/rfc4361}{RFC~4361},
    \href{http://tools.ietf.org/html/rfc5494}{RFC~5494},
    \href{http://tools.ietf.org/html/rfc6221}{RFC~6221},
    \href{http://tools.ietf.org/html/rfc6422}{RFC~6422} og
    \href{http://tools.ietf.org/html/rfc6644}{RFC~6644}
  \item Overf�ring av IPv6-pakker over Ethernet:
    \href{http://tools.ietf.org/html/rfc2464}{RFC~2464} og
    \href{http://tools.ietf.org/html/rfc6085}{RFC~6085}
  \end{itemize}
\end{frame}

\begin{frame}
  \frametitle{Kort om IPv6}
  \framesubtitle{RFC-er om IPv6}
  \pause
  \begin{itemize}[<+->]
  \item Adressearkitektur:
    \href{http://tools.ietf.org/html/rfc4291}{RFC~4291},
    \href{http://tools.ietf.org/html/rfc5952}{RFC~5952} og
    \href{http://tools.ietf.org/html/rfc6052}{RFC~6052}
  \item Unicastadresser:
    \href{http://tools.ietf.org/html/rfc3587}{RFC~3587}
  \item ULA: \href{http://tools.ietf.org/html/rfc4193}{RFC~4193}
  \item Autokonfigurering av adresser:
    \href{http://tools.ietf.org/html/rfc4862}{RFC~4862}
  \item Random interface ID:
    \href{http://tools.ietf.org/html/rfc4941}{RFC~4941}
  \item Prefiks-baserte multicastadresser:
    \href{http://tools.ietf.org/html/rfc3306}{RFC~3306},
    \href{http://tools.ietf.org/html/rfc3956}{RFC~3956} og
    \href{http://tools.ietf.org/html/rfc4489}{RFC~4489}
  \item For programmerere av nettverksprogrammer:
    \href{http://tools.ietf.org/html/rfc4038}{RFC~4038}
  \end{itemize}
\end{frame}

\section{IPv6-header}
\begin{frame}
  \frametitle{IPv6-header}
  \pause
  \begin{itemize}[<+->]
  \item Bla, bla, bla
  \end{itemize}
\end{frame}

\section{Adresser}
\begin{frame}
  \frametitle{Adresser}
  \pause
  \begin{itemize}[<+->]
  \item 128 bit
  \item Heksadesimal notasjon
  \item 16 bit grupperes, adskilt med kolon
  \item Ledende nuller kan sl�yfes
  \item To eller flere 16-bit-blokker med nuller kan sl�s sammen til
    \texttt{::} (dobbelkolon), bare �n gang pr.\ adresse
  \item Prefikslengde angis ved � slenge p� en skr�strek og antall
    signifikante bit fra venstre mot h�yre
  \end{itemize}
\end{frame}

\subsection{Adressedemo}
\begin{frame}
  \frametitle{Adresser}
  \framesubtitle{Adressedemo}
  \pause
  \begin{itemize}[<+->]
  \item Uninett:\\\texttt{2001:0700:0000:0000:0000:0000:0000:0000\phantom{/32}}
  \item FSI:\\\texttt{2001:0700:1100:0000:0000:0000:0000:0000\phantom{/48}}
  \item IT-avdelingen@FSI:\\\texttt{2001:0700:1100:0003:0000:0000:0000:0000\phantom{/64}}
  \item Tronds D531:\\\texttt{2001:0700:1100:0003:0221:70FF:FE73:686E\phantom{/128}}
  \end{itemize}
\end{frame}

\begin{frame}
  \frametitle{Adresser}
  \framesubtitle{Adressedemo}
  %\pause
  \begin{itemize}%[<+->]
  \item Uninett:\\\texttt{\alert{2001:0700}:0000:0000:0000:0000:0000:0000\phantom{/32}}
  \item FSI:\\\texttt{\alert{2001:0700:1100}:0000:0000:0000:0000:0000\phantom{/48}}
  \item IT-avdelingen@FSI:\\\texttt{\alert{2001:0700:1100:0003}:0000:0000:0000:0000\phantom{/64}}
  \item Tronds D531:\\\texttt{\alert{2001:0700:1100:0003:0221:70FF:FE73:686E}\phantom{/128}}
  \end{itemize}
\end{frame}

\begin{frame}
  \frametitle{Adresser}
  \framesubtitle{Adressedemo}
  %\pause
  \begin{itemize}%[<+->]
  \item Uninett:\\\texttt{2001:0700:0000:0000:0000:0000:0000:0000\phantom{/32}}
  \item FSI:\\\texttt{2001:0700:1100:0000:0000:0000:0000:0000\phantom{/48}}
  \item IT-avdelingen@FSI:\\\texttt{2001:0700:1100:0003:0000:0000:0000:0000\phantom{/64}}
  \item Tronds D531:\\\texttt{2001:0700:1100:0003:0221:70FF:FE73:686E\phantom{/128}}
  \end{itemize}
\end{frame}

\begin{frame}
  \frametitle{Adresser}
  \framesubtitle{Adressedemo}
  %\pause
  \begin{itemize}%[<+->]
  \item Uninett:\\\texttt{2001:\alert{0}700:\alert{000}0:\alert{000}0:\alert{000}0:\alert{000}0:\alert{000}0:\alert{000}0\phantom{/32}}
  \item FSI:\\\texttt{2001:\alert{0}700:1100:\alert{000}0:\alert{000}0:\alert{000}0:\alert{000}0:\alert{000}0\phantom{/48}}
  \item IT-avdelingen@FSI:\\\texttt{2001:\alert{0}700:1100:\alert{000}3:\alert{000}0:\alert{000}0:\alert{000}0:\alert{000}0\phantom{/64}}
  \item Tronds D531:\\\texttt{2001:\alert{0}700:1100:\alert{000}3:\alert{0}221:70FF:FE73:686E\phantom{/128}}
  \end{itemize}
\end{frame}

\begin{frame}
  \frametitle{Adresser}
  \framesubtitle{Adressedemo}
  %\pause
  \begin{itemize}%[<+->]
  \item Uninett:\\\texttt{2001:\alert{700}:\alert{0}:\alert{0}:\alert{0}:\alert{0}:\alert{0}:\alert{0}\phantom{/32}}
  \item FSI:\\\texttt{2001:\alert{700}:1100:\alert{0}:\alert{0}:\alert{0}:\alert{0}:\alert{0}\phantom{/48}}
  \item IT-avdelingen@FSI:\\\texttt{2001:\alert{700}:1100:\alert{3}:\alert{0}:\alert{0}:\alert{0}:\alert{0}\phantom{/64}}
  \item Tronds D531:\\\texttt{2001:\alert{700}:1100:\alert{3}:\alert{221}:70FF:FE73:686E\phantom{/128}}
  \end{itemize}
\end{frame}

\begin{frame}
  \frametitle{Adresser}
  \framesubtitle{Adressedemo}
  %\pause
  \begin{itemize}%[<+->]
  \item Uninett:\\\texttt{2001:700:0:0:0:0:0:0\phantom{/32}}
  \item FSI:\\\texttt{2001:700:1100:0:0:0:0:0\phantom{/48}}
  \item IT-avdelingen@FSI:\\\texttt{2001:700:1100:3:0:0:0:0\phantom{/64}}
  \item Tronds D531:\\\texttt{2001:700:1100:3:221:70FF:FE73:686E\phantom{/128}}
  \end{itemize}
\end{frame}

\begin{frame}
  \frametitle{Adresser}
  \framesubtitle{Adressedemo}
  %\pause
  \begin{itemize}%[<+->]
  \item Uninett:\\\texttt{2001:700:\alert{0:0:0:0:0:0}\phantom{/32}}
  \item FSI:\\\texttt{2001:700:1100:\alert{0:0:0:0:0}\phantom{/48}}
  \item IT-avdelingen@FSI:\\\texttt{2001:700:1100:3:\alert{0:0:0:0}\phantom{/64}}
  \item Tronds D531:\\\texttt{2001:700:1100:3:221:70FF:FE73:686E\phantom{/128}}
  \end{itemize}
\end{frame}

\begin{frame}
  \frametitle{Adresser}
  \framesubtitle{Adressedemo}
  %\pause
  \begin{itemize}%[<+->]
  \item Uninett:\\\texttt{2001:700\alert{::}\phantom{/32}}
  \item FSI:\\\texttt{2001:700:1100\alert{::}\phantom{/48}}
  \item IT-avdelingen@FSI:\\\texttt{2001:700:1100:3\alert{::}\phantom{/64}}
  \item Tronds D531:\\\texttt{2001:700:1100:3:221:70FF:FE73:686E\phantom{/128}}
  \end{itemize}
\end{frame}

\begin{frame}
  \frametitle{Adresser}
  \framesubtitle{Adressedemo}
  %\pause
  \begin{itemize}%[<+->]
  \item Uninett:\\\texttt{2001:700::\phantom{/32}}
  \item FSI:\\\texttt{2001:700:1100::\phantom{/48}}
  \item IT-avdelingen@FSI:\\\texttt{2001:700:1100:3::\phantom{/64}}
  \item Tronds D531:\\\texttt{2001:700:1100:3:221:70FF:FE73:686E\phantom{/128}}
  \end{itemize}
\end{frame}

\begin{frame}
  \frametitle{Adresser}
  \framesubtitle{Adressedemo}
  %\pause
  \begin{itemize}%[<+->]
  \item Uninett:\\\texttt{2001:700::\alert{/32}}
  \item FSI:\\\texttt{2001:700:1100::\alert{/48}}
  \item IT-avdelingen@FSI:\\\texttt{2001:700:1100:3::\alert{/64}}
  \item Tronds D531:\\\texttt{2001:700:1100:3:221:70FF:FE73:686E\alert{/128}}
  \end{itemize}
\end{frame}

\begin{frame}
  \frametitle{Adresser}
  \framesubtitle{Adressedemo}
  %\pause
  \begin{itemize}%[<+->]
  \item Uninett:\\\texttt{2001:700::/32}
  \item FSI:\\\texttt{2001:700:1100::/48}
  \item IT-avdelingen@FSI:\\\texttt{2001:700:1100:3::/64}
  \item Tronds D531:\\\texttt{2001:700:1100:3:221:70FF:FE73:686E/128}
  \end{itemize}
\end{frame}

\subsection{Adressetyper}
\begin{frame}
  \frametitle{Adresser}
  \framesubtitle{Adressetyper}
  \pause
  \begin{itemize}[<+->]
  \item Det finnes flere adressetyper med forskjellige bruksomr�der:
    \begin{itemize}[<+->]
    \item Link-local-adresser
    \item Site-local-adresser
    \item Offentlige unicast-adresser
    \item Unike, lokale, aggregerbare adresser
    \item Anycast-adresser
    \item Multicast-adresser
    \end{itemize}
  \item Adressene best�r som oftest av 2 deler:
    \begin{itemize}[<+->]
    \item Prefiks
    \item Grensesnittadresse
    \end{itemize}
  \item Unntak gjelder for multicastadressene, som best�r av flere
    deler
  \item Merk at broadcast er avskaffet og er erstattet i stor grad med
    link-local-multicast
  \end{itemize}
\end{frame}

%\subsection{Link-local-adresser}
\begin{frame}
  \frametitle{Adresser}
  \framesubtitle{Link-local-adresser}
  \pause
  \begin{itemize}[<+->]
  \item Definert: \href{http://tools.ietf.org/html/rfc4291}{RFC~4291}
  \item Bruksomr�de: lokal kommunikasjon internt i VLAN-et, sentral
    for autokonfigurasjon, blir ikke videresendt til andre VLAN eller
    til internett
  \item Prefiks: \texttt{FE80::/10}
  \item De 54 neste bitene skal settes til null
  \item De siste 64 bitene settes til MAC48-adressa omformet til modda EUI64-format
  \item Eksempel: \texttt{FE80::221:70FF:FE73:686E}
  \end{itemize}
\end{frame}

%\subsection{Site-local-adresser}
\begin{frame}
  \frametitle{Adresser}
  \framesubtitle{Site-local-adresser}
  \pause
  \begin{itemize}[<+->]
  \item Definert: \href{http://tools.ietf.org/html/rfc3513}{RFC~3513}
  \item Bruksomr�de: privat, intern kommunikasjon p� lik linje med
    \href{http://tools.ietf.org/html/rfc1918}{RFC~1918}
  \item Prefiks: \texttt{FEC0::/10}
  \item De 38 neste bitene settes til null
  \item De 16 neste bitene kan brukes til subnet-ID
  \item De siste 64 bitene kan settes til MAC48-adressa omformet til
    modda EUI64-format eller settes manuelt
  \item Eksempel: \texttt{FEC0::DEAD:BEEF:1337}
  \item Ikke bruk site-local-adresser (\href{http://tools.ietf.org/html/rfc3879}{RFC~3879})
  \item Site-local-adresser er erstatta med ULA (\href{http://tools.ietf.org/html/rfc4193}{RFC~4193})
  \end{itemize}
\end{frame}

\end{document}

% % % % % % % % % % % % % % % % % % % % % % % % % % % % % % % % % % %

\section{}
\begin{frame}
  \frametitle{}
  \pause
  \begin{itemize}[<+->]
  \item Bla, bla, bla
  \end{itemize}
\end{frame}

\section{}
\begin{frame}
  \frametitle{}
  \pause
  \begin{itemize}[<+->]
  \item Bla, bla, bla
  \end{itemize}
\end{frame}

\section{}
\begin{frame}
  \frametitle{}
  \pause
  \begin{itemize}[<+->]
  \item Bla, bla, bla
  \end{itemize}
\end{frame}

\section{}
\begin{frame}
  \frametitle{}
  \pause
  \begin{itemize}[<+->]
  \item Bla, bla, bla
  \end{itemize}
\end{frame}

\section{}
\begin{frame}
  \frametitle{}
  \pause
  \begin{itemize}[<+->]
  \item Bla, bla, bla
  \end{itemize}
\end{frame}

\section{Sammendrag}
\begin{frame}
  \frametitle{Sammendrag}
  \pause
  \begin{itemize}[<+->]
  \item Bla, bla, bla
  \end{itemize}
\end{frame}

\section{Emne 1}
\begin{frame}
  \frametitle{Emne 1}
  \pause
  \begin{itemize}[<+->]
  \item Bla, bla, bla
  \end{itemize}
\end{frame}

\subsection{Underemne 1a}
\begin{frame}
  \frametitle{Emne 1}
  \framesubtitle{Underemne 1a}
  \pause
  \begin{itemize}[<+->]
  \item Bla, bla, bla
  \end{itemize}
\end{frame}

Local Variables:
TeX-PDF-mode:t
End:
